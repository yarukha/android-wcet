\documentclass{beamer}
\usepackage{amsmath}
\usepackage{stmaryrd}
\usepackage{graphicx}   
\graphicspath{ {./pics/}}
%Information to be included in the title page:
\title{Profiling energy consumption of Android applications}
\author{Paul Robert}
\date{2022}

\begin{document}

\frame{\titlepage}

\begin{frame}
    \frametitle{la honte...}
    \includegraphics{lahonte}
\end{frame}

\begin{frame}
    \frametitle{Quick summary}
    \begin{itemize}
        \pause
        \item What is an energy consuming application? \pause
        \item Few words on the Android bytecode \pause
        \item Control flow graph \pause
        \item ILP to solve the problem \pause
        \item Refine the analysis with abstract interpretation
    \end{itemize}
\end{frame}

\begin{frame}
    \frametitle{Definitions and specification}
    \begin{itemize}
        \pause
        \item Trying to identify spikes of energy consumption: WCEC in a given time \pause
        \item Simple model: we assume energy and time for native Android operations is known
        \item APK= Dalvik/ART bytecode files + data \pause 
        \item bytecode = list of methods declaration
        \item ~240 bytecode instruction + 1000+ native methods
    \end{itemize}
\end{frame}

\begin{frame}
    \frametitle{Method CFG}
    We can easily build a CFG for all the methods defined in a bytecode \pause 
    \begin{figure}[h]
        \includegraphics[scale=0.21]{cfg.jpg}
    \end{figure}
\end{frame}

\begin{frame}
    \frametitle{Linking the methods}
    \pause
    \begin{itemize}
        \item Each method has its CFG, defined independantly \pause 
        \item Methods are linked through invokes \pause
        \item Native methods can be and are invoked \pause
    \end{itemize}
    We may compute the consumption and execution time of each block for each cfg \\ \pause 
    "All" we have to do is to find the WCEC path in a given time 
\end{frame}

\begin{frame}
    \frametitle{Reduction to ILP}
    We may reduce our path problem to an ILP one \pause
    \begin{align*}
        & \text{maximize}   && \mathbf{c}^\mathrm{T} \mathbf{x}\\
        & \text{subject to} && A \mathbf{x} \le \mathbf{b}, \\
        &  && \mathbf{x} \ge \mathbf{0}, \\
        & \text{and} && \mathbf{x} \in \mathbb{Z}^n,
        \end{align*}
    Where $\mathbf x$ is the execution count of each block\\ \pause 
    A solution may be infeasible in pratice, we have to refine the analysis...
\end{frame}

\begin{frame}
    \frametitle{Refining the analysis}
    \pause 
    \begin{itemize}
        \item We need to add more constraints to the ILP problem \pause 
        \item Infering bounds to block execution count from each method CFG \pause 
        \item We will use AI \pause 
    \end{itemize}
    \break
    \break
    Equational semantics \pause $\rightarrow$ abstraction \pause $\rightarrow$ fixpoint solver 
\end{frame}

\begin{frame}
    \frametitle{(Very) short Abstract interpretation summary}
    \pause 
    \begin{itemize}
        \item At certain program points $\in V$, we want to map concrete values to abstract ones \pause
        \item The invariants of a program are defined by the fixpoint of a (monotonic) function $f: (V \rightarrow  C) \rightarrow (V \rightarrow  C)$ \pause 
        \item Need to lose details to obtain computability \pause
        \item We have a monotonic abstraction function $\alpha : C \rightarrow A$ \pause 
        \item Abstract invariants: fixpoint of $f': (V \rightarrow  A) \rightarrow (V \rightarrow  A)$ \pause 
    \end{itemize}
\end{frame}

\begin{frame}
    \frametitle{Example of collecting semantics}
    \pause 
    Given a set of invariant equations for program points $\mathcal X_1 \cdots \mathcal X_n$: \\
    $\mathcal X_i = \mathcal F_i(\mathcal X_1,\cdots, \mathcal X_n)$\\ \pause
    The function $F(X)=(\mathcal F_1 (X),\cdots, \mathcal F_n (X))$ can be used to compute a fixpoint \pause 
    \begin{figure}[h]
        \includegraphics[scale=0.2]{code.png} \pause \includegraphics[scale=0.3]{eqs_graph.png} \pause \includegraphics[scale=0.2]{equations.png}
    \end{figure}
\end{frame}

\begin{frame}
    \frametitle{Abstracting}
    \begin{itemize}
        \pause
        \item Define an abstract domain (a lattice) \pause 
        \item Simple ones: Constant, Sign, Interval \pause
        \item Carefully define abstraction function and abstract operators \pause 
        \item Compute a fixpoint $\rightarrow$ done!
    \end{itemize}
\end{frame}
\begin{frame}
    \frametitle{Back to the project}
    \begin{itemize}
        \item We want to know more about block execution counts \pause 
        \item Analyze each method individually \pause 
        \item Interval domain $I$ for each block \pause 
        \item Finite number of register used in a method, abstracted to $D$ \pause 
        \item Abstraction on both $D$ and $I$
    \end{itemize}
\end{frame}

\begin{frame}
    \frametitle{The Octagon Domain}
    \begin{itemize}
        \pause
        \item We need to know constraints in-between registers \pause
        \item Intervals are not enough \pause 
        \item Octagon: $V_i \pm V_j \leq c$
    \end{itemize}
\end{frame}
\end{document}