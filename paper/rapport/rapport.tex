\documentclass{article}
\bibliographystyle{unsrt}
\usepackage{graphicx}
\usepackage{hyperref}
\usepackage{url}
\graphicspath{ {./} }
\usepackage[utf8]{inputenc}
%\usepackage[margin=2cm]{geometry}
\usepackage[margin=1cm]{geometry}
\newgeometry{margin=1cm,bottom=2cm}
\usepackage{amsmath}
\usepackage{amssymb}

\title{Energetic profiling of Android applications}
\author{Paul Robert}
\date{2022}
\begin{document}
\maketitle

\begin{abstract}
    Techniques to determine the worst case of time or enery con
\end{abstract}
    
\section*{Blabla about the project}

\section*{Blabla about the state of the art of WCEC and WCET}

A lot of work has been put on statically obtaining precise WCET \cite{wilhelm_abstract_2008}, however this work deals with precise estimate and assume that the behaviour of the hardware is known. 
These work obtain good results and take into account cache and pipeline.\\
obtaining precise results is not the objective of our project and we cannot assume anything about hardware behaviour. In particular, we will not deal at all with cache miss, pipeline stalls or multi threaded executions.\\
We will assume a sequential execution and work with execution bounds, which is critized in \cite{wilhelm_real_2020}. 


\section*{Describtion of the Bytecode}
A dissassembled Android bytecode is a list of methods, each of these methods being a block of bytecode instructions.
There is a total of 247 instructions, described in \cite{noauthor_dalvik_nodate}.\\
With this bytecode, we can use the techniques described in \cite{zhao_analyzing_1999} to build the CFGs of all the methods.\\
The nodes of a CFG are blocks of instructions rather than unique instructions, allowing for better representation of the graph.\\
The edges represents the branchings inside the methods and can be of 3 types: \emph{Regular}, \emph{Taken} and \emph{Exceptions}.\\\\
For $n$ methods, we then obtain $n$ CFGs that can be analyzed separately, each method can call other methods through \emph{invoke} instructions. The methods called can be either the methods described in the bytecode or one of the 4000+ Android methods.\\
These last methods are the ones making hardware calls and are the biggest source of energy consumption.\\\\
The VM has a finite number of registers, in practice, each method use a few dozen of registers 

\section*{Quick words on the power model and the "consuming definition"}

A power model is tricky both to define and establish. Due to the portative nature of the Android Virtual Machine, it is impossible to define on for all hardwares it can run onto.
Instead we can define one for at least one device and infer results from it.\\
We can first define a function $P_i$ from the sets of AVM instructions $I$ to a domain $D$. \\
This set could take many forms but to keep things as simple as possible, $D$ can just be $\mathbb N\times\mathbb N$, and the instruction power model is a function returning the consumption bound of each instruction.\\
We can define a similar function $P_{ma}$ for the Android methods.\\\\
A possible definition for an application $a_1$ to be more consuming than an app $a_2$ is: \\
For a given time $t$, the worst case energy consumption (WCEC) for a time $t$ is higher in $a_1$ than in $a_2$.\\
Note that this worst case energy consumption can be achieved starting with any method defined in the application bytecode. Namely, we try to measur which application has the biggest energy consumption spike.
\section*{Structure of the analyzer}

The analysis for a given time $t$ is divided into several parts:
\begin{enumerate}
    \item analyse each methods to obtain bounds for its energy consumption and execution time, as well as the method calls possibles.
    \item use ILP to determine which possible list of method calls is the WCEC in the time $t$.
\end{enumerate}

\subsection*{Method analysis}
The analysis of the CFG $C=(V,E)$, where the set of nodes $V$ of size are instructions blocks, is meant to produce bounds for the number of execution of each blocks.\\
Abstract analysis is used to achieve that.\\
We then use these bounds and the properties of flow conservation in the CFG to generate an ILP problem, as described in \cite{li_performance_nodate}.\\
maximize $c^\intercal x$ with $Ax \leq b, x \geq 0$ and $x\in \mathbb{Z}^n$\\
$x$ is the vector of the number of execution of the  $n= \lvert V \rvert $ blocks, $c$ the energy consumed by each block, and $b$ the execution count bound of the blocks.\\\\
After solving this ILP problem, we may know time and energy consumed bounds, as well as list of methods called.\\
\emph{Obviously needs to be refined}
\subsection*{Linking the methods}
Once we have analysed all $N$ methods, we may try to find the execution section that reach the WCEC in time $t$.\\
Several techniques may be used to obtain this result, knowing that the analysis has to be done for the $N$ methods as start of the execution section.\\
We can either 
\begin{enumerate}
    \item explore the graph to find the WCEC, very costly in practice
    \item generate another ILP problem for the execution section starting with a method $M_i$ and ending with all the methods reachable in the time $t$.
\end{enumerate} 
 
\section*{Abstract interpretation for methods analysis}
We use the theory of abstract interpretation to obtain bounds on the execution count of each instructions blocks of a method $M$.\\
The analysis is done on the method CFG $C$ built using the techniques defined earlier.\\
For each node, we abstract both the value of each register as well as the number of times this block is executed.\\
The number of registers $n_r$ used in the method can be determined beforehand. The abstract domain $D_R$ used for the value of the registers is yet to be defined. A simple interval domain might not suffice.\\
An interval domain $\mathbb{I}$ should however suffice for the execution count.\\
The abstraction is thus done on a product domain of  the $n_r$ $D_R$ domains and $\mathbb{I}$.\\
Each register is 32 bits wide and support every standard and bitwise operations.\\ 


\nocite{*}
\bibliography{lib.bib}
\end{document}